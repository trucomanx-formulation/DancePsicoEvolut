\documentclass[a4paper,10pt]{article}
\usepackage[utf8]{inputenc}
\usepackage{comment}

\usepackage[
backend=biber,
style=alphabetic,
sorting=ynt
]{biblatex}
\addbibresource{bibliography.bib}   %>>>> bibliography data in bibliography.bib
%\bibliography{bibliography}   %>>>> bibliography data in bibliography.bib
%\bibliographystyle{spiebib}   %>>>> makes bibtex use spiebib.bst


%%%%%%%%%%%%%%%%%%%%%%%%%%%%%%%%%%%%%%%%%%%%%%%%%%%%%%%%%%%%%%%%%%%%%%%%%%%%%%%%
\usepackage{url}
%
\usepackage[breaklinks]{hyperref}
\usepackage{breakurl}

%%%%%%%%%%%%%%%%%%%%%%%%%%%%%%%%%%%%%%%%%%%%%%%%%%%%%%%%%%%%%%%%%%%%%%%%%%%%%%%%
\usepackage{graphicx}
\usepackage{subcaption}
%%%%%%%%%%%%%%%%%%%%%%%%%%%%%%%%%%%%%%%%%%%%%%%%%%%%%%%%%%%%%%%%%%%%%%%%%%%%%%%%

\usepackage[svgnames]{xcolor} % Enabling colors by their 'svgnames'

\usepackage{amsmath}
\usepackage{amsfonts}
\usepackage{amssymb}


%%%%%%%%%%%%%%%%%%%%%%%%%%%%%%%%%%%%%%%%%%%%%%%%%%%%%%%%%%%%%%%%%%%%%%%%%%%%%%%%
\usepackage{amsthm}
\newtheorem{mydef}{Definição}

%%%%%%%%%%%%%%%%%%%%%%%%%%%%%%%%%%%%%%%%%%%%%%%%%%%%%%%%%%%%%%%%%%%%%%%%%%%%%%%%

\usepackage[top=25mm, bottom=25mm, left=25mm, right=25mm]{geometry}

%%%%%%%%%%%%%%%%%%%%%%%%%%%%%%%%%%%%%%%%%%%%%%%%%%%%%%%%%%%%%%%%%%%%%%%%%%%%%%%%

%opening
\title{Paradigmas na Dança, Estereotipos, Neurosexismo e Psicologia Evolutiva}
\author{Fernando Pujaico Rivera, -------- -------- --------}
\date{}
%%%%%%%%%%%%%%%%%%%%%%%%%%%%%%%%%%%%%%%%%%%%%%%%%%%%%%%%%%%%%%%%%%%%%%%%%%%%%%%%
%%%%%%%%%%%%%%%%%%%%%%%%%%%%%%%%%%%%%%%%%%%%%%%%%%%%%%%%%%%%%%%%%%%%%%%%%%%%%%%%
%%%%%%%%%%%%%%%%%%%%%%%%%%%%%%%%%%%%%%%%%%%%%%%%%%%%%%%%%%%%%%%%%%%%%%%%%%%%%%%%
%%%%%%%%%%%%%%%%%%%%%%%%%%%%%%%%%%%%%%%%%%%%%%%%%%%%%%%%%%%%%%%%%%%%%%%%%%%%%%%%
\begin{document}

\maketitle

\begin{abstract}

Este artigo busca mostrar que os roles de genero na dança,
não são um producto do sexismo da sociedade atual, 
e sim estereotipos que são um reflexo das tendencias 
reproduivas e psico-evolutivas do ser humano.

Para este efeito abordaremos temas como:

\begin{itemize}
\item A paradoja nórdica: Nos paises com maior nivel de igualdad social, 
de genero e economica, é onde mais marcado se vem o roles de genero,
emquanto que paises com maior desiguladad em geral,
os roles de genero nao estão tão diferenciados.

\item Steven Pinker, os estereotipos são mais o resultado que a causa.

\item  Como costumava mencionar Gustav Mahler, resgatar 
a tradição ``é a transmissão do fogo e não 
a adoracão das cinzas''\footnote{Hurtándole el tiempo al tiempo https://revistas.ucr.ac.cr/index.php/escena/article/view/27470/27645}.

\item Estereotipos são equivalentes à predição linear. Exemplo, aeropuerto, 
pessoas com medias de colore, optimo seria revisar, seguinda a relação 
conhecidad com as cores dos sapatos, mesmo nao conhecendo o vinculo,
error seria fixar esta tendencia mesmo conhecendo novos dados e relaçoes.

\item Esrategias e mecanismos evolutivos para a reprodução dependendo do sexo do individuo.

\end{itemize}
Finalmente concluimos que os estereotipos são muito validos na função que cumprem, 
pois são criados por uma tendencia intrinseca na sociedade,
que procura um objetivo comum para perpetuarse no tempo. 
O problema não é a existencia destos estereotipos, 
que deben ser considerados uma tendencia e não uma regra;
o problema é que a existemcia de um estereotipo, 
nos leve equivocadamente a asumir que todo mundo tem que entrar dentro de algum desses modelos,
e que são normas onde nada pode escapar.
Assim, se estabelece que a raiz do conflito não é o estereotipo, 
e sim a polarização de uma postura, 
que leva à obrigatoriedade em cumprir um rol estereotipado.
De modo que, se deduz que o problema é a obrigação onde deberia existir a escolha.

\end{abstract}


%%%%%%%%%%%%%%%%%%%%%%%%%%%%%%%%%%%%%%%%%%%%%%%%%%%%%%%%%%%%%%%%%%%%%%%%%%%%%%%%
%%%%%%%%%%%%%%%%%%%%%%%%%%%%%%%%%%%%%%%%%%%%%%%%%%%%%%%%%%%%%%%%%%%%%%%%%%%%%%%%
%%%%%%%%%%%%%%%%%%%%%%%%%%%%%%%%%%%%%%%%%%%%%%%%%%%%%%%%%%%%%%%%%%%%%%%%%%%%%%%%
\section{Paradigma na dança}


%%%%%%%%%%%%%%%%%%%%%%%%%%%%%%%%%%%%%%%%%%%%%%%%%%%%%%%%%%%%%%%%%%%%%%%%%%%%%
\section{Agradecimentos}
Agradecimentos especial a ``la entropia de Vale'' 
NEUROSEXISMO - ¿Son nuestros cerebros diferentes PARTE 2
%\begi
%%%%%%%%%%%%%%%%%%%%%%%%%%%%%%%%%%%%%%%%%%%%%%%%%%%%%%%%%%%%%%%%%%%%%%%%%%%%%%%%
%%%%%%%%%%%%%%%%%%%%%%%%%%%%%%%%%%%%%%%%%%%%%%%%%%%%%%%%%%%%%%%%%%%%%%%%%%%%%%%%
%%%%%%%%%%%%%%%%%%%%%%%%%%%%%%%%%%%%%%%%%%%%%%%%%%%%%%%%%%%%%%%%%%%%%%%%%%%%%%%%

\section{comentario 1}
Nesse paradigma na dança um individuo envia informação e outro recebe, e ninguém esta proibido de tomar o papel de condutor. No modelo de dança mencionado existe um estereotipo, sim, porem isso não indica um caráter machista do modelo, pois nesse paradigma alguém tem que enviar informação e outro receber, para tomar cada um desses papeis se usou o contexto cultural da criação das danças, sendo que nessas épocas menos liberais o contato entre pessoas estava mais restringido, e a dança cumpria alem do papel social um medio de relacionamentos, formando assim uma metáfora de algo que se ve muito na natureza, o cortejo do macho à fêmea, como muitos animais não humanos fazem, e isso não da o adjetivo ideológico de machista a esses animais, eles só cumprem o estereotipo básico na natureza, mostrar suas melhores qualidades para ter a oportunidade de se reproduzir, entre essas qualidades geralmente procuradas no macho, na natureza, estão: a força, a liderança, as cores e os movimentos rituais. No caso das fêmeas que são geralmente as que escolhem ao macho, as caraterísticas que alguns machos procuram na fêmea são: empatia, sensibilidade, doçura, compaixão, tolerância, boa contextura física para ter filhos etc. Por isso que inconsciente ou conscientemente, e sem filtros ideológicos estes papeis foram atribuídos. Porem um estereotipo, não é uma norma com caráter de lei, é só um comportamento em media que cumpre um grupo de indivíduos, assim o estereotipo, é um fim, não um causa. O problema não é a existência destes estereótipos, que devem ser considerados uma tendencia e não uma regra.

\section{comentario 2}

\begin{comment}
eu concordo que a idéia de uma pessoa conduzir outra na dança, apesar de autoritária, não necessariamente é machista, mas acontece que nesse planeta a condução ainda é, em regra, monopólio dos homens. Estou entendendo que você não vê machismo na estrutura da condução tradicional ("cavalheiro" conduz e "dama" responde), pois entende que homem conduzir não é uma regra mas uma "tendência" imitada dos bichos tentando acasalar, é isso? Se for isso eu tenho que discordar de você. A convenção de que o homem deve conduzir a mulher surgiu no período cavalheiresco em que se acreditava que o homem deveria tutelar a mulher pois teoricamente elas eram incapazes de decidirem o rumo de suas vidas de forma autônoma, e seguiu nessa lógica até hoje. É importante atentarmos para questões históricas para não cairmos em achismos. Agora, a nossa opinião não tem tanta relevância diante da opinião de milhares de mulheres da comunidade da dança de salão que estão há um bom tempo denunciando esse machismo. Acho que devemos respeitar essas mulheres e ouví-las, afinal, não somo animais irracionais cortejando, somos seres humanos conscientes. E se conduzir não é machista, desprezar a opinião das mulheres certamente é. Então vamos ouvir mais elas? Vai ser bom para abrir a cabeça, garanto. Abração.
\end{comment}
Não, a condução em exclusividade a uma pessoa em função do seu sexo, não é regra; é tendencia, que é diferente. Poderia ser visto como regra se alguma pessoa estivesse proibida por seu sexo, e isto não acontece, miguem vai ser punido, nê existe organização a nível governamental que regulamente, assim se a alternativa não esta proibida então está permitida. A pergunta correta não é se existe sexismo em este ato; e sim, se este ato está permitido, porque as pessoas de ambos sexos não tomam este papel com igual frequência?. 

Concordo sim em que entre os fatores que influenciaram, nessa época, a que uma pessoa determinada por seu sexo, tivesse o papel de condutor, existiu o que agora chamaríamos de sexismo, porem não estou seguro se essa palavra existia na época, pois os roles de gênero nesses tempos estavam bem definidos. Mas estes estereótipos na dança perduraram muito depois a que todas as pessoas ganharemos igualdade ante a lei, independentemente do sexo, raça, etc, o que mostra que o sexismo que podia existir na época, não era o único fator determinante para fazer perdurar estes estereótipos, mesmo na época atual, que estos temas são discutidos, e todas as pessoas ganhamos igualdade ante a lei, não existe uma paridade na escolha do papel do condutor na dança. E existe escolha porque a alternativa não está proibida. Assim esse sexismo não existe, pois as pessoas não estão punidas por tomar esse papel, mesmo não existindo a dança compartilhada, o problema da paridade na escolha da condução seria resolvido simplesmente si esses milhares de pessoas tomaram a decisão de conduzir. Por conseguinte a dança compartilhada não é indispensável para cumprir esse proposito, a solução não passa pela dança compartilhada e sim pela escolha do papel de condutor.A dança compartilhada tem um papel importante como estudo de novos paradigma da comunicação na dança, abrindo mais possibilidades técnicas. Eu não estou em contra da dança compartilhada, eu sinto muita curiosidade por conhecer toda a riqueza técnica que está ainda por ser revelada. Porem, meus argumentos vão em contra de afirmar que o paradigma da condução é um ato sexista, e que esses novos paradigma na dança vem a resolver esse problema. Como argumentei anteriormente, a solução a paridade na condução passa simplesmente por uma escolha, sim se implementa uma dança compartilhada pensando que a paridade surgirá espontaneamente, simplesmente volveremos a ver estes estereótipos baixo as regras deste novo paradigma; pois não entenderíamos, que pessoas de ambos sexos temos diferencias de comportamento, marcadas como especie como qualquer outro animal, e que florescem quando agimos naturalmente. Para mas informação desta tendencia natural posso citar o "Paradoxo nórdico" sobre a escolha faculdades e temas de estudo entre as pessoas de ambos sexos.

\section{comentario 3}
\begin{comment}
 o brazilian zouk dance council nos puniu por fazermos condução compartilhada, essa organização não governamental serve? kkkkkk ai ai Enfim, pelo que entendi você acha que não existe machismo na dança de salão e que a condução compartilhada não serve para aquilo que ela se propõe a resolver. E então porque inventaram essa coisa, seria mimimi? Você também sugeriu que caso existisse machismo, a solução seria simplesmente a pessoa conduzida pedir para inverter os papeis de condução porque isso é bem fácil de acontecer. E mais uma vez você repetiu que acredita que o ser humano não decide sua própria cultura, é o instinto animal que forma tudo. Ou seja, você é a favor desses papéis de gênero e acha que são naturais. É isso, entendi corretamente?
\end{comment}

\begin{comment}
Fiquei interessado no seu caso no congresso, gostaria souber que tipo de punição receberam, quanto tempo durou a punição, e qual foi o motivo que argumentaram para aplicar em vocês. 
\end{comment}
Eu não acredito que exista machismo na sociedade como todo, acho que o termo correto é que existe sexismo em algumas pessoas, não na sociedade, o termo machismo implica que o responsável do estado atual, é só um setor da sociedade só por seu sexo, e a esse tipo de afirmações se lhe conhece atualmente como sexismo. Eu acredito que a dança compartilhada é um ótimo objeto de estudo, mas argumento que não vem a solucionar nenhum estereotipo de conduta social, sim se aplicara agora em seu totalidade, ainda veríamos marcados os mesmos estereótipos porem nesse novo médio de comunicação. Sobre inverter os papeis de condução: não falo de tomar o papel do outro, pois essa afirmação parte da ideia que algo não me corresponde, simplesmente afirmo que se debe tomar o papel de condutor como uma escolha, não pedir o papel pra ver se é dado, cada pessoa que deseja conduzir simplesmente, procura um par e pede a pessoa pra acompanhar numa dança (com prévio aprendizado da técnica obviamente). Não afirmo que o ser humano não decide, argumento que as pessoas como qualquer outro animal temos uma tendencia natural (set default), não regra, em tomar algumas decisões, porem como qualquer outro animal, podemos decidir segui-las ou não. Mas temos sim, alguns tipos de comportamento marcados como padrão de inicio, que com o tempo se manifestam socialmente como estereótipos, mas isto não quer dizer que nada escapa a este padrão inicial, e sim que é o ponto de partida desde onde nosso mundo racional inicia a trabalhar. Porem, seguir estes padroẽs ou estereótipos não nos vira sexistas, é simplesmente nossa naturaleza inicial, sem nenhuma tara ideológica como qualquer outro animal. Eu estou a favor da livre escolha das pessoas de tomar o papel que desejam, mas argumento em contra das pessoas que afirmam, que as pessoas que seguem ou desejam seguir algum destes padrões, indicam uma conduta sexista, quando na verdade é uma manifestação da sua própria natureza.

\section{Conclusões}
Nas nossas pesquisas sobre novos paradigmas da condução, seja compartilhando, 
co-conduzindo ou produzindo a dança de forma mutua, 
tenho achado muito material não 
acadêmico\footnote{Postagens em blogs com temas relativos a dança.} 
focando a importância destes paradigmas na dança, 
numa luta social sobre o papel das pessoas na dança seguindo o sexo.
Acho que é um error promover ou divulgar estes novos paradigmas na dança 
como um método para a liberação da opressão do condutor escolhido por seu sexo, 
isso só é uma postura ideológica. 
Se só isso fosse o problema, 
bastaria com empoderar as pessoas que atualmente não usam, 
ou se planteiam usar, esse papel na sociedade, 
a tomar ou pedir o papel de condutor nas danças, 
dado que atualmente não existe nenhuma norma que o proíba,
ou alguma instituição que regule ou sancione este comportamento,
e que exija em caráter mandatório o papel de um individuo na condução seguindo o sexo,
isso ate agora era só uma convenção social, 
dado que o único método de comunicação conhecido na dança era a comunicação unidirecional de emissor fixo;
consequentemente um individuo devia tomar o papel de condutor e outro de seguidor;
de modo que foi usado o padrão cultural da época, da criação das danças, 
para escolher o papel de cada pessoa na dança a dois. 
Assim,  atualmente o papel de condutor está habilitado para toda pessoa que se interesse em 
cultivá-lo e aprendê-lo. 
Pelo qual se promovemos o valor destes novos paradigmas na dança baseando-nos na luta social e sexual, 
estos não teriam sentido de existir, 
pois nenhuma pessoa na sociedade atual está proibida de tomar o papel de condutor.


%%%%%%%%%%%%%%%%%%%%%%%%%%%%%%%%%%%%%%%%%%%%%%%%%%%%%%%%%%%%%%%%%%%%%%%%%%%%%%%%
%%%%%%%%%%%%%%%%%%%%%%%%%%%%%%%%%%%%%%%%%%%%%%%%%%%%%%%%%%%%%%%%%%%%%%%%%%%%%%%%
%%%%%%%%%%%%%%%%%%%%%%%%%%%%%%%%%%%%%%%%%%%%%%%%%%%%%%%%%%%%%%%%%%%%%%%%%%%%%%%%
%%%%%%%%%%%%%%%%%%%%%%%%%%%%%%%%%%%%%%%%%%%%%%%%%%%%%%%%%%%%%%%%%%%%%%%%%%%%%%%%

\medskip
 
\printbibliography

\end{document}
