\documentclass[a4paper,10pt]{article}
\usepackage[utf8]{inputenc}
\usepackage{comment}

\usepackage[
backend=biber,
style=alphabetic,
sorting=ynt
]{biblatex}
\addbibresource{bibliography.bib}   %>>>> bibliography data in bibliography.bib
%\bibliography{bibliography}   %>>>> bibliography data in bibliography.bib
%\bibliographystyle{spiebib}   %>>>> makes bibtex use spiebib.bst


%%%%%%%%%%%%%%%%%%%%%%%%%%%%%%%%%%%%%%%%%%%%%%%%%%%%%%%%%%%%%%%%%%%%%%%%%%%%%%%%
\usepackage{url}
%
\usepackage[breaklinks]{hyperref}
\usepackage{breakurl}

%%%%%%%%%%%%%%%%%%%%%%%%%%%%%%%%%%%%%%%%%%%%%%%%%%%%%%%%%%%%%%%%%%%%%%%%%%%%%%%%
\usepackage{graphicx}
\usepackage{subcaption}
%%%%%%%%%%%%%%%%%%%%%%%%%%%%%%%%%%%%%%%%%%%%%%%%%%%%%%%%%%%%%%%%%%%%%%%%%%%%%%%%

\usepackage[svgnames]{xcolor} % Enabling colors by their 'svgnames'

\usepackage{amsmath}
\usepackage{amsfonts}
\usepackage{amssymb}


%%%%%%%%%%%%%%%%%%%%%%%%%%%%%%%%%%%%%%%%%%%%%%%%%%%%%%%%%%%%%%%%%%%%%%%%%%%%%%%%
\usepackage{amsthm}
\newtheorem{mydef}{Definição}

%%%%%%%%%%%%%%%%%%%%%%%%%%%%%%%%%%%%%%%%%%%%%%%%%%%%%%%%%%%%%%%%%%%%%%%%%%%%%%%%

%opening
\title{Paradigmas na Dança, Neurosexismo e Psicologia Evolutiva}
\author{Fernando Pujaico Rivera, -------- -------- --------}

%%%%%%%%%%%%%%%%%%%%%%%%%%%%%%%%%%%%%%%%%%%%%%%%%%%%%%%%%%%%%%%%%%%%%%%%%%%%%%%%
%%%%%%%%%%%%%%%%%%%%%%%%%%%%%%%%%%%%%%%%%%%%%%%%%%%%%%%%%%%%%%%%%%%%%%%%%%%%%%%%
%%%%%%%%%%%%%%%%%%%%%%%%%%%%%%%%%%%%%%%%%%%%%%%%%%%%%%%%%%%%%%%%%%%%%%%%%%%%%%%%
%%%%%%%%%%%%%%%%%%%%%%%%%%%%%%%%%%%%%%%%%%%%%%%%%%%%%%%%%%%%%%%%%%%%%%%%%%%%%%%%
\begin{document}

\maketitle

\begin{abstract}

Este artigo busca mostrar que os roles de genero na dança,
não são um producto do sexismo cultural, 
e sim estereotipos que são um reflexo das tendencias socias, 
reproduivas e evolutivas do ser humano.

Para este efeito abordaremos temas como:

\begin{itemize}
\item A paradoja nórdica: Nos paises com maior nivel de igualdad social, 
de genero e economica, é onde mais marcado se vem o roles de genero,
emquanto que paises com maior desiguladad em geral,
os roles de genero nao estão tão diferenciados.

\item Steven Pinker, os estereotipos são mais o resultado que a causa.

\item Estereotipos equivalente predição linear. Exemplo, aeropuerto, 
pessoas com medias de colore, optimo seria revisar, seguinda a relação 
conhecidad com as cores dos sapatos, mesmo nao conhecendo o vinculo,
error seria fixar esta tendencia mesmo conhecendo novos dados e relaçoes.

\item Esrategias e mecanismos evolutivos para a reprodução seguindo o sexo.

\end{itemize}
Finalmente concluimos que os estereotipos são muito validos na função que cumprem, 
pois são criados por uma tendencia intrinseca na sociedade,
que procura um objetivo comum para perpetuarse no tempo. 
O problema não é a existencia destos estereotipos, 
que deben ser considerados uma tendencia e não uma regra;
o problema é que a existemcia de um estereotipo, 
nos leve equivocadamente a asumir que todo mundo tem que entrar dentro de algum desses modelos,
e que são normas onde nada pode escapar.
Assim, se estabelece que a raiz do conflito não é o estereotipo, 
e sim a polarização de uma postura, 
que leva à obrigatoriedade em cumprir um rol estereotipado.
De modo que, se deduz que o problema é a obrigação onde deberia existir a escolha.

\end{abstract}


%%%%%%%%%%%%%%%%%%%%%%%%%%%%%%%%%%%%%%%%%%%%%%%%%%%%%%%%%%%%%%%%%%%%%%%%%%%%%%%%
%%%%%%%%%%%%%%%%%%%%%%%%%%%%%%%%%%%%%%%%%%%%%%%%%%%%%%%%%%%%%%%%%%%%%%%%%%%%%%%%
%%%%%%%%%%%%%%%%%%%%%%%%%%%%%%%%%%%%%%%%%%%%%%%%%%%%%%%%%%%%%%%%%%%%%%%%%%%%%%%%
\section{Paradigma na dança}


%%%%%%%%%%%%%%%%%%%%%%%%%%%%%%%%%%%%%%%%%%%%%%%%%%%%%%%%%%%%%%%%%%%%%%%%%%%%%
\section{Agradecimentos}
Agradecimentos especial a ``la entropia de Vale'' 
NEUROSEXISMO - ¿Son nuestros cerebros diferentes PARTE 2
%\begi
%%%%%%%%%%%%%%%%%%%%%%%%%%%%%%%%%%%%%%%%%%%%%%%%%%%%%%%%%%%%%%%%%%%%%%%%%%%%%%%%
%%%%%%%%%%%%%%%%%%%%%%%%%%%%%%%%%%%%%%%%%%%%%%%%%%%%%%%%%%%%%%%%%%%%%%%%%%%%%%%%
%%%%%%%%%%%%%%%%%%%%%%%%%%%%%%%%%%%%%%%%%%%%%%%%%%%%%%%%%%%%%%%%%%%%%%%%%%%%%%%%
\section{Conclusões}
Nas nossas pesquisas sobre novos paradigmas da condução, seja compartilhando, 
co-conduzindo ou produzindo a dança de forma mutua, 
tenho achado muito material não 
acadêmico\footnote{Postagens em blogs com temas relativos a dança.} 
focando a importância destes paradigmas na dança, 
numa luta social sobre o papel das pessoas na dança seguindo o sexo.
Acho que é um error promover ou divulgar estes novos paradigmas na dança 
como um método para a liberação da opressão do condutor escolhido por seu sexo, 
isso só é uma postura ideológica. 
Se só isso fosse o problema, 
bastaria com empoderar as pessoas que atualmente não usam, 
ou se planteiam usar, esse papel na sociedade, 
a tomar ou pedir o papel de condutor nas danças, 
dado que atualmente não existe nenhuma norma que o proíba,
ou alguma instituição que regule ou sancione este comportamento,
e que exija em caráter mandatório o papel de um individuo na condução seguindo o sexo,
isso ate agora era só uma convenção social, 
dado que o único método de comunicação conhecido na dança era a comunicação unidirecional de emissor fixo;
consequentemente um individuo devia tomar o papel de condutor e outro de seguidor;
de modo que foi usado o padrão cultural da época, da criação das danças, 
para escolher o papel de cada pessoa na dança a dois. 
Assim,  atualmente o papel de condutor está habilitado para toda pessoa que se interesse em 
cultivá-lo e aprendê-lo. 
Pelo qual se promovemos o valor destes novos paradigmas na dança baseando-nos na luta social e sexual, 
estos não teriam sentido de existir, 
pois nenhuma pessoa na sociedade atual está proibida de tomar o papel de condutor.


%%%%%%%%%%%%%%%%%%%%%%%%%%%%%%%%%%%%%%%%%%%%%%%%%%%%%%%%%%%%%%%%%%%%%%%%%%%%%%%%
%%%%%%%%%%%%%%%%%%%%%%%%%%%%%%%%%%%%%%%%%%%%%%%%%%%%%%%%%%%%%%%%%%%%%%%%%%%%%%%%
%%%%%%%%%%%%%%%%%%%%%%%%%%%%%%%%%%%%%%%%%%%%%%%%%%%%%%%%%%%%%%%%%%%%%%%%%%%%%%%%
%%%%%%%%%%%%%%%%%%%%%%%%%%%%%%%%%%%%%%%%%%%%%%%%%%%%%%%%%%%%%%%%%%%%%%%%%%%%%%%%

\medskip
 
\printbibliography

\end{document}
